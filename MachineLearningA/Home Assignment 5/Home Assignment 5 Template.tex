\documentclass[a4paper,12pt]{article}
\usepackage{etoolbox}
\newtoggle{exam}
\togglefalse{exam}
\newcommand{\an}{4}

\usepackage{../assignment}
\usepackage{framed}
\usepackage{color}
\usepackage{amsfonts}
\usepackage{amsmath}
\usepackage{amssymb}
\usepackage{mathrsfs}
\usepackage{wrapfig}

\usepackage{xcolor}  % For a colorfull presentation
\usepackage{listings}  % For presenting code 

\usepackage{hyperref}

% Definition of a style for code, matter of taste
\lstdefinestyle{mystyle}{
  language=Python,
  basicstyle=\ttfamily\footnotesize,
  backgroundcolor=\color[HTML]{F7F7F7},
  rulecolor=\color[HTML]{EEEEEE},
  identifierstyle=\color[HTML]{24292E},
  emphstyle=\color[HTML]{005CC5},
  keywordstyle=\color[HTML]{D73A49},
  commentstyle=\color[HTML]{6A737D},
  stringstyle=\color[HTML]{032F62},
  emph={@property,self,range,True,False},
  morekeywords={super,with,as,lambda},
  literate=%
    {+}{{{\color[HTML]{D73A49}+}}}1
    {-}{{{\color[HTML]{D73A49}-}}}1
    {*}{{{\color[HTML]{D73A49}*}}}1
    {/}{{{\color[HTML]{D73A49}/}}}1
    {=}{{{\color[HTML]{D73A49}=}}}1
    {/=}{{{\color[HTML]{D73A49}=}}}1,
  breakatwhitespace=false,
  breaklines=true,
  captionpos=b,
  keepspaces=true,
  numbers=none,
  showspaces=false,
  showstringspaces=false,
  showtabs=false,
  tabsize=4,
  frame=single,
}
\lstset{style=mystyle}


\begin{document}
\title{Machine Learning A\\Home Assignment 5}
\author{\color{red}Your name}
\date{}
\maketitle


\section{Principal Component Analysis}

\subsection{PCA and preprocessing}
\subsubsection{By the book}
\paragraph{b)}
In question part b), we were asked to 

\paragraph{c)}
In question part c), we were asked to 


\subsubsection{In the center}
Let $\vec{S}$ be a  $d \times d$  matrix.

\subsection{PCA in practice}

\subsubsection{Explained variance}
\begin{figure}
    \centering
    \caption{Eigenspectrum of the handwritten digits data.}
    \label{fig:my_figure}
\end{figure}

The  explained variance is shown in Fig.~\ref{fig:my_figure}.

Ten components are  \dots to explain 80\% of the variance.
This can be seen from \dots


\section{Logistic Regression in PyTorch}

\subsection{Logistic regression in PyTorch}
First, I completed the definition of the logistic regression model:
\begin{lstlisting}
class LogisticRegressionPytorch(nn.Module):
    def __init__(self, d, m):
        super(LogisticRegressionPytorch, self).__init__()
        # LAYER DEFINITION MISSING
    def forward(self, x):
        # RETURN VALUE MISSING
\end{lstlisting}
Then 

\end{document}