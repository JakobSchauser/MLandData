\documentclass[a4paper,12pt]{article}

\usepackage{a4wide}
\usepackage{amsfonts}
\usepackage{amsmath}
\usepackage{amssymb}


\usepackage{graphicx}  % For including images

\usepackage{xcolor}  % For a colorfull presentation
\usepackage{listings}  % For presenting code 

\usepackage{hyperref}

% Definition of a style for code, matter of taste
\lstdefinestyle{mystyle}{
  language=Python,
  basicstyle=\ttfamily\footnotesize,
  backgroundcolor=\color[HTML]{F7F7F7},
  rulecolor=\color[HTML]{EEEEEE},
  identifierstyle=\color[HTML]{24292E},
  emphstyle=\color[HTML]{005CC5},
  keywordstyle=\color[HTML]{D73A49},
  commentstyle=\color[HTML]{6A737D},
  stringstyle=\color[HTML]{032F62},
  emph={@property,self,range,True,False},
  morekeywords={super,with,as,lambda},
  literate=%
    {+}{{{\color[HTML]{D73A49}+}}}1
    {-}{{{\color[HTML]{D73A49}-}}}1
    {*}{{{\color[HTML]{D73A49}*}}}1
    {/}{{{\color[HTML]{D73A49}/}}}1
    {=}{{{\color[HTML]{D73A49}=}}}1
    {/=}{{{\color[HTML]{D73A49}=}}}1,
  breakatwhitespace=false,
  breaklines=true,
  captionpos=b,
  keepspaces=true,
  numbers=none,
  showspaces=false,
  showstringspaces=false,
  showtabs=false,
  tabsize=4,
  frame=single,
}
\lstset{style=mystyle}

\begin{document}
\title{Machine Learning A\\Home Assignment 1}
\author{\color{red}Your name}
\date{}
\maketitle



\section{Make Your Own}



\section{Digits  Classification  with $K$ Nearest  Neigh-bors}




\section{Linear Regression}
\subsection{Implementation of linear regression}\label{sec:lr_imp}
I implemented linear regression using the pseudoinverse as introduced
in the lecture. My implementation can be found in the notebook
\texttt{Assignment1Question6MyName.ipynb}.

I computed the parameters of a regression model in the following way:
\begin{lstlisting}
 def important_function:
     # vital code
\end{lstlisting}



\begin{figure}
  \begin{center}
%    \includegraphics[width=.75\textwidth]{Assignment1_Question6_Plot1}
  \end{center}
  \caption{This figure shows the PCB concentration vs.{} the age of
    the fish in years.
    Every figure and table has a caption and is referred to 
    in the text body.\label{fig:q6p1}}
\end{figure}
\subsection{Building the first model}
Figure~\ref{fig:q6p1} shows the loaded data.
To fit a model of the form
\begin{equation*}
 h(x) = \exp(ax+b) 
 \end{equation*}
 with parameters $a,b\in\mathbb R$, I transformed the output data (the
 $y$ values) by applying the natural logarithm first:
\begin{lstlisting}
 new = magic_transform(old)
\end{lstlisting}
The I used the algorithm described in Section~\ref{sec:lr_imp}
to model $\ln y$ by $ax+b$.
I found $a=\dots$ and $b=\dots$.

\noindent\dots 
\subsection{\dots}
\dots 

\bibliography{bibliography}  % If you have some references, use BibTeX

\end{document}
